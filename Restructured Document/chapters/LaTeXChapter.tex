This chapter will explain how to install, configure, and use LaTeX Editors. This guide is written in LaTeX. LaTeX (pronouced lay-tek) is a high-quality typesetting system, similar to Markdown. 
\section{Installing a TeX Distribution}
	\subsection{MiKTeX}
	\subsection{LaTeX Tools}
	\subsection{TeX Live}
\section{Editors}
	\subsection{Using Sublime}
	\subsection{Using Atom}
	\subsection{Using Visual Studio Code}
		Visual Studio Code has well-supported LaTeX editor packages which provide a strong feature set while retaining the benefits of VS Code's integration with Git as well as Git-based packages. This section of the guide assumes that you have VS Code installed and understand how to navigate the menus and install packages. It also assumes that you have installed one of the TeX distributions from the previous section.
		\begin{enumerate}
			\item First, install LaTeX language support for VS Code. This provides coloring and style guides for VS Code.
			\item Next, install the LaTeX Workshop package from James Yu. This is the primary LaTeX package for VS Code.
			\item Read through the documentation for LaTeX Workshop and configure settings as necessary. This guide will focus on configuration using TeX Live.
			\item Install other packages to enhance the features of LaTeX. Some recomendations are listed below.
		\end{enumerate}
		\begin{itemize}
			\item LaTeX Utilities: Formatted pastes for unicode characters, table cells, images, and csv files.
			\item LaTeX Preview
			\item LaTeX ...
		\end{itemize}
	
\section{Other LaTeX Systems}
	\subsection{LyX}
	\subsection{Overleaf}
	