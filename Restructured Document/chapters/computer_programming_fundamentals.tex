Computer programming may seem like a daunting field for beginner programmers. It does not need to be. At its core, computer programming is providing a set of instructions to a computer. Each line of code contains one or more instructions. Sometimes instructions interfere with each other and cause errors. Imagine telling someone who has never seen a sandwich before how to make a peanut butter and jelly sandwich. The steps may be:
\begin{enumerate}
	\item Get Bread, Peanut Butter, Jelly, Knife, and Plate.
	\item Place plate.
	\item Place two pieces of bread side by side on plate. Call them Slice A and Slice B.
	\item Put peanut butter on knife.
	\item Use knife to spread peanut butter on Slice A.
	\item Clean knife.
	\item Put jelly on knife.
	\item Use knife to spread jelly on Slice B.
	\item Combine Slice A with Slice B by placing Slice B on Slice A.
	\item Eat Sandwich.
	\item Clean Knife
	\item Clean Plate
\end{enumerate}

Provided no external changes, this set of instructions will always lead to a proper peanut butter and jelly sandwich. However, some people may argue about if you should use grape jelly or strawberry jelly, chunky or smooth peanut butter. Some people are adamant that you must put Jelly on Slice A, never peanut butter. Some argue that if one starts with a loaf of bread instead of slices then it produces a better sandwich. These options represent the different ways that people program. There are many different sets of instructions that produce peanut butter and jelly sandwiches. Programming is very smiliar. There are many different ways to code a program that will lead to the same result.

However, there are ways which are more efficient than others, and there are ways to make code easier to understand. When writing code, it is important to write with a consistent programming style that allows others to understand your code. Code must be maintained over time, and may transition between programmers. This chapter should serve to introduce the basics of computer programming as a platform for the Cyber Truck Experience program. Computer programming is too large of a topic to cover in any meaningful depth in this guide, so this guide will primarily focus on explaining which programming languages our group mainly uses, as well as reasons for their use. There may also be recommended resources for learning more.

\section{Data Types}

\section{Introduction to Computer Programming Languages}
	Programming languages are the dialects of computer instructions that are written. While manyconcepts carry over between languages, differences in sytax (how you write and format theinstructions) creates situations where functionally equivalent commands may appear completelydifferent.
	For example, to get the porgram to ``say'' something or show a string of text as an output, youmay use some kind of ``print'' command. ``Printing'' in computer programming doesn't cause adocument to be printed from your printer, but instead it ``prints'' something on the screen or ina log that may be viewed by humans. The ``print'' command varies between languages, and evendifferent versions of the same language. A key difference between Python 2 and Python 3 is thesyntax of print statements. In Python 2 a print statement looks like:
	\begin{lstlisting}[language=python, autogobble=true]
       	print "Hello World!"
       \end{lstlisting}
       However, in Python 3 the command includes parentheses:
       \begin{lstlisting}[language=python, autogobble=true]
       	print("Hello World!")
       \end{lstlisting}
       Using the wrong print command will cause the code to crash, even though they are very similar.Small mistakes such as omitting one of the parentheses can also cause the code to crash. It isimportant to pay attention to the spacing and punctuation. Other languages have similar commands
       Java: 
       \begin{lstlisting}[language=java, autogobble=true]
       	println("Hello World!")
       \end{lstlisting}
       Arduino:
       \begin{lstlisting}[language=c, autogobble=true]
       	Serial.println("Hello World!")
    \end{lstlisting}
    Even though all of these commands are slightly different, they all cause the same result. Theconsole will output the string ``Hello World!''
\section{Programming with Python 3}
  	An important tool in any hacker's toolkit is the ability to interact with collected data. In this section, we will be using a programming language called ``Python."
   	\subsection{Background}
   		Python is a general purpose, versatile, and popular programming/scripting language. it is great as a first language because it is concise and easy to read. It is also a good language to have in any programer's stack as it can be used for everything from web development to software development and data science applications.
   	\subsection{CodeAcademy}
   		Learning a new programming language, especially as a first language, can have a sharp learning curve. The website \href{https://www.codecademy.com/}{CodeAcademy.com} has an excellent free beginners Python course. Rather than attempt to replicate such a course, this guide simply recommends that you complete CodeAcademy python course to learn ptyhon. While CodeAcademy offeres a paid service with extra features and courses, it is NOT necessary to make any purchases.

   		While completing the course take notes over funtions that may be useful (importing/exporting files, searching through lists, data structures, etc.)
   	\subsection{Python Editors and IDEs}
   	\subsection{Threading}
   	\subsection{Serial Communication with PySerial}
   	\subsection{Graphing with MatPlotLib}
   	\subsection{Making GUIs with PyQt5}
   	\subsection{Jupyter Notebooks}

\section{Programming with Arduino}

\section{Programming with C}

